%-------------------------------------------------------------------------------
%	SECTION TITLE
%-------------------------------------------------------------------------------
\cvsection{Projects}


%-------------------------------------------------------------------------------
%	CONTENT
%-------------------------------------------------------------------------------
\begin{cventries}
%TODO
%---------------------------------------------------------
  \cventry
    {University of Washington} % Organisation
    {Automaticity in the reading circuitry} % Project
    {Seattle, WA} % Location
    {2015 - 2019} % Date(s) chktex 8
    {
      \begin{cvitems} % Description(s) of project
        \item {We measured magnetoencephalography in school aged children (N = 42, 7–12 years of age) to measure examine word-selective brain responses under multiple experimental conditions. The results show that even in the presence of overt distraction, strong word-selective responses were found in select language regions. Critically, this automatic word-selective response was indicative of reading skill: the strength of word-selective responses correlated with individual reading skill.}
        \item {\textbf{Technical Skills:} Data Engineering, Python, Digital Signal Processing, Data Modeling.}
        \item {\textbf{Soft Skills:} Time management, Teamwork, Technical Writing.}
      \end{cvitems}
    }

%---------------------------------------------------------
  \cventry
    {University of Washington} % Organisation
    {Effectively combining temporal projection noise suppression methods in magnetoencephalography} % Project
    {Seattle, WA} % Location
    {2018 - 2020} % Date(s) chktex 8
    {
      \begin{cvitems} % Description(s) of project
        \item {This project examined the effects of temporal signal space separation (tSSS) and oversampled temporal projection (OTP) on noise suppression in low SNR magnetoencephalography data. The findings apply to clinical populations such as epilepsy, single-trial data, or cases of sparse data.}
        \item {\textbf{Technical Skills:} Data Visualization, Python, Digital Signal Processing, Data Modeling.}
        \item {\textbf{Soft Skills:} Mentoring, Teamwork, Scientific Writing.}
      \end{cvitems}
    }

%---------------------------------------------------------
  \cventry
    {University of Washington} % Organisation
    {Mne-Bids: Organizing Electrophysiological Data into the Bids Format and Facilitating Their Analysis} % Project
    {Seattle, WA} % Location
    {2018 - 2019} % Date(s) chktex 8
    {
      \begin{cvitems} % Description(s) of project
        \item {MNE-BIDS links BIDS and MNE-Python to speed up analyses, develop more reliable code, and facilitate sharing of data and code with co-workers and collaborators. BIDS, the Brain Imaging Data Structure, is a standard that describes how to organize neuroimaging and electrophysiological data. MNE-Python is an open-source Python package for exploring, visualizing, and analyzing human neurophysiological data such as MEG, EEG, sEEG, ECoG, and more. It includes data input/output modules, preprocessing, visualization, source estimation, time-frequency analysis, connectivity analysis, machine learning, and statistics.}
        \item {\textbf{Technical Skills:} Python Software Development, Git, Github}
        \item {\textbf{Soft Skills:} Teamwork}
      \end{cvitems}
    }

%---------------------------------------------------------
  \cventry
    {University of Washington} % Organisation
    {Using magnetoencephalography to examine word recognition, lateralization, and future language skills in 14-month-old infants} % Project
    {Seattle, WA} % Location
    {2014 - 2019} % Date(s) chktex 8
    {
      \begin{cvitems} % Description(s) of project
        \item {I developed an acquisition protocol and managed data collection for word learning in young infants (N = 27, 39–42 weeks gestational age) using magnetoencephalography (MEG) during a spoken word recognition experimental paradigm. Used Python software tools for digital signal processing and 3D modeling to examine the relationship between brain responses and prospective measures of vocabulary growth. The findings were discussed in terms of theory on cerebral lateralization and individual differences related to attention that play an essential role in language learning.}
        \item {\textbf{Technical Skills:} Python, Digital Signal Processing, Statistical Modeling}
        \item {\textbf{Soft Skills:} Teamwork, Project Management, Data Management}
      \end{cvitems}
    }

%---------------------------------------------------------
  \cventry
    {The Children's Hospital of Philadelphia} % Organisation
    {Auditory Magnetic Mismatch Field Latency: A Biomarker for Language Impairment in Autism} % Project
    {Philadelphia, PA} % Location
    {2010 - 2011} % Date(s) chktex 8
    {
      \begin{cvitems} % Description(s) of project
        \item {I applied mixed-linear modeling to data for children with ASD (N = 51, 6–15 years of age) who underwent neuropsychological evaluation, including tests of language function and magnetoencephalographic (MEG) recording during speech discrimination. Features in the MEG timeseries data were used to operationalize a biomarker for neuronal speech discrimination in response to stimulation. Mixed-linear modeling ANOVA revealed significantly slower (p < .001) discrimination in children with ASD than in control subjects. Receiver operator characteristic analysis indicated a sensitivity of 82.4\% and specificity of 71.2\% for diagnosing language impairment based on MEG feature engineering.}
        \item {\textbf{Technical Skills:} Mixed Linear Modeling, Analysis of Variance, SPSS, Receiver Operator Curve Analysis, Case Control}
        \item {\textbf{Soft Skills:} Scientific Writing}
      \end{cvitems}
    }

%---------------------------------------------------------
\end{cventries}
