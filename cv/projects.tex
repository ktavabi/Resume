%-------------------------------------------------------------------------------
%	SECTION TITLE
%-------------------------------------------------------------------------------
\cvsection{Projects}


%-------------------------------------------------------------------------------
%	CONTENT
%-------------------------------------------------------------------------------
\begin{cventries}
%TODO
%---------------------------------------------------------
  \cventry
    {University of Washington} % Organisation
    {Automaticity in the reading circuitry} % Project
    {Seattle, WA} % Location
    {2015 - 2019} % Date(s) chktex 8
    {
      \begin{cvitems} % Description(s) of project
        \item {Measured brain activity in school-aged children (N = 42, 7–12 years of age) with magnetoencephalography to examine word-selective brain responses during reading.}
        \item {Developed data acquisition procedures and PYTHON routines for digital signal processing, dimensionality reduction (PCA), data transformations, and 3D statistical modeling of dense-array timeseries data.}
      \end{cvitems}
    }

%---------------------------------------------------------
  \cventry
    {University of Washington} % Organisation
    {Effectively combining temporal projection noise suppression methods in magnetoencephalography} % Project
    {Seattle, WA} % Location
    {2018 - 2020} % Date(s) chktex 8
    {
      \begin{cvitems} % Description(s) of project
        \item {Mentored a graduate student with data visualization, analysis, and manuscript preparation for a study describing the efficacy of various noise sub-space projection methods for preprocessing dense-array electrophysiology data before 3D statistical modeling.}
      \end{cvitems}
    }

%---------------------------------------------------------
  \cventry
    {University of Washington} % Organisation
    {Mne-Bids: Organizing Electrophysiological Data into the Bids Format and Facilitating Their Analysis} % Project
    {Seattle, WA} % Location
    {2018 - 2019} % Date(s) chktex 8
    {
      \begin{cvitems} % Description(s) of project
        \item {Collaborated with an international team of software engineers to create open-source Python applications to speed up analyses, enhance code reliability, and facilitate data and code sharing amongst co-workers and collaborators.}
      \end{cvitems}
    }

%---------------------------------------------------------
  \cventry
    {University of Washington} % Organisation
    {Using magnetoencephalography to examine word recognition, lateralization, and future language skills in 14-month-old infants} % Project
    {Seattle, WA} % Location
    {2014 - 2019} % Date(s) chktex 8
    {
      \begin{cvitems} % Description(s) of project
        \item {Investigated early childhood language learning by combining neuropsychological measurements and experimental word discrimination paradigm in a cohort of typically developing infants (N = 27, 39–42 weeks old).}
        \item {Developed data acquisition procedures and built PYTHON routines for digital signal processing, data mining, feature engineering, and regression model to assess the relationship between neuropsychological and prospective behavioral performance measurements of vocabulary growth.}
      \end{cvitems}
    }

%---------------------------------------------------------
  \cventry
    {The Children's Hospital of Philadelphia} % Organisation
    {Auditory Magnetic Mismatch Field Latency: A Biomarker for Language Impairment in Autism} % Project
    {Philadelphia, PA} % Location
    {2010 - 2011} % Date(s) chktex 8
    {
      \begin{cvitems} % Description(s) of project
        \item {Leveraged nonparametric linear mixed modeling to overhaul statistical analysis of a large dataset containing neuropsychological measurements of speech discrimination in children diagnosed with autism spectrum disorders (N = 51, 6–15 years of age) and used receiver operator characteristic analysis to characterize diagnostic sensitivity and specificity for language impairment based on neuropsychological measurements.}
      \end{cvitems}
    }

%---------------------------------------------------------
\end{cventries}
