%-------------------------------------------------------------------------------
% SECTION TITLE
%-------------------------------------------------------------------------------
\cvsection{Work Experience}

%-------------------------------------------------------------------------------
% CONTENT
%-------------------------------------------------------------------------------
\begin{cventries}

    %---------------------------------------------------------
    \cventry
    {Core Competencies}
    {Various Organizations}
    {Domestic and International Locations}
    {2011 -- Present}
    {
      \begin{cvitems}
        \item \textbf{Machine Learning \& Statistical Analysis:} Random Forest, Logistic Regression, Statistical Learning, Feature Engineering, TF-IDF Vectorization, Classification Algorithms, Dimensionality Reduction (PCA), Time-Series Analysis, ROC Analysis, A/B Testing, Bayesian Statistics
        \item \textbf{Biological Data Systems:} Digital Signal Processing, Neurophysiological Data Processing, Multi-sensor Data Fusion, Clinical Data Analysis, Spectral-Temporal Analysis, Behavioral Data Modeling
        \item \textbf{Technical Infrastructure:} Python (NumPy, Pandas, SciPy, Scikit-learn), R (ggplot2, lme4, Tidyverse), MATLAB, SQL, Git/GitHub, Cloud Computing, Data Pipeline Development, Scalable ML Systems
        \item \textbf{Research Leadership:} Grant Management (\$2M+), Cross-functional Team Leadership, Open Science Advocacy, Technical Mentoring, Scientific Communication, Business Intelligence Management
      \end{cvitems}
    }
  %---------------------------------------------------------

  %---------------------------------------------------------
  \cventry
    {Principal Data Scientist}
    {Washington State Department of Corrections}
    {Tumwater, WA}
    {May 2023 -- Present}
    {
      \begin{cvitems}
        \item \textbf{Business Intelligence \& Analytics Leadership}
        \item Business Intelligence Manager overseeing team of 5 analysts, managing data analytics projects using agile methodologies and Git workflow for version control and collaborative development
        \item Developed ML model using TF-IDF vectorization of sparse text data and logistic classifiers to analyze behavioral data with millions of cases for nearly 30K individuals
        \item Built scalable data science infrastructure using Python, SQL databases, and cloud computing to process longitudinal datasets for real-time decision support systems
        \item Applied advanced statistical methods including causal inference, A/B testing, and time-series analysis to measure program effectiveness
        \item Consulted with agency executives and cross-functional business units on data strategy, translating complex analytics into actionable business insights for strategic decision-making
      \end{cvitems}
    }
  %---------------------------------------------------------
  \cventry
    {Research Science Engineer}
    {Institute for Learning \& Brain Sciences, University of Washington}
    {Seattle, WA}
    {2011 -- 2023}
    {
      \begin{cvitems}
        \item \textbf{Machine Learning for Neuroscience Applications}
        \item Designed and implemented ML models for speech and language processing research, analyzing neural responses using Random Forest, logistic classifier, and advanced digital signal processing techniques
        \item Developed analytical procedures for digital signal processing and data harmonization to process high-dimensional neurophysiological datasets from 500+ participants across multiple research projects
        \item Collaborated on development of MNE-BIDS and MNE-Python, internationally-adopted open-source platforms for electrophysiological data analysis, creating standardized workflows adopted by research teams globally
        \item Applied machine learning techniques including PCA, time-frequency analysis, and pattern recognition to extract biological insights from speech-related neural signals in pediatric populations
        \item Managed interdisciplinary collaborations with biostatisticians, clinicians, and engineers to translate ML research into practical neuroscience applications
      \end{cvitems}
    }
  %---------------------------------------------------------
  \cventry
    {Post Doctoral Researcher}
    {The Children's Hospital of Philadelphia, Department of Radiology}
    {Philadelphia, PA}
    {2008 -- 2011}
    {
      \begin{cvitems}
        \item \textbf{Translational Biomarker Development}
        \item Developed statistical classification systems using nonparametric linear mixed modeling and ROC analysis to identify neural biomarkers for language impairment in autism spectrum disorders (N=51, 6-15 years)
        \item Implemented advanced signal processing algorithms for noise reduction and feature extraction from clinical neurophysiological data, achieving diagnostic sensitivity and specificity suitable for clinical translation
        \item Applied statistical modeling and longitudinal analysis techniques to understand developmental trajectories in pediatric neurological disorders
        \item Earned NIH Loan Repayment Award for outstanding translational biomedical research bridging statistical methods with clinical applications
      \end{cvitems}
    }
  %---------------------------------------------------------
  \cventry
    {Visiting Research Scientist}
    {MRC Cognition and Brain Sciences Unit}
    {Cambridge, UK}
    {2007 -- 2008}
    {
      \begin{cvitems}
        \item \textbf{Neural Electrophysiology}
        \item Conducted research on neural mechanisms of speech perception using computational modeling approaches
        \item Developed experimental paradigms investigating neural plasticity using advanced statistical inference methods
        \item Collaborated with leading European neuroscience teams on large-scale neuroimaging studies
      \end{cvitems}
    }

  %---------------------------------------------------------
\end{cventries}
