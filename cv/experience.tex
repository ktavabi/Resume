%-------------------------------------------------------------------------------
%	SECTION TITLE
%-------------------------------------------------------------------------------
\cvsection{Work Experience}


%-------------------------------------------------------------------------------
%	CONTENT
%-------------------------------------------------------------------------------
\begin{cventries}

  %---------------------------------------------------------
  \cventry
  {Research Science Engineer} % Job title
  {University of Washington} % Organisation
  {Seattle, WA} % Location
  {Nov 2011 - Dec 2022} % Date(s) chktex 8
  {
    \begin{cvitems} % Description(s) of tasks/responsibilities
      \item {I Joined the \href{https://ilabs.uw.edu/}{ILABS} MEG Brain Imaging lab as a postdoctoral researcher and developed collection, aggregation, exploratory analysis, and small-data visualization workflows for human health data (HIPPA), resulting in publications on terabytes of data for 100+ participants in various scientific journals (\href{https://doi.org/10.1111/desc.12427}{i}, \href{https://doi.org/10.1016/j.dcn.2020.100901}{ii}, \href{https://doi.org/10.1121/1.5137262}{iii}).}
      \item {Since 2013 I have contributed to open-source computing software tools for small data mining or `fishing,' automation \& collaborative workflows, and statistical modeling of human neurophysiological data (MEG, EEG). \href{https://github.com/LABSN/mnefun}{MNEFUN} is a small-scale, Python wrapper application used to deploy data science solutions for individual MEEG data at ILABS.}
      \item {Contributed to software application for digital signal processing and analysis of individual health data geared at faster coding implementation, more robust analysis, and code sharing between co-workers and collaborations at the institution level \href{https://mne.tools/mne-bids/}{MNE-bids}.}
      \item {MNE-bids codebase was featured in a \href{https://joss.theoj.org/papers/10.21105/joss.01896}{Journal of Open Source Software publication} and is actively maintained by a vibrant open-source developer community.}
      \item {In 2014 I was promoted to UW staff scientist (RSE-IV) after developing a series of biomedical imaging data science projects to study language learning in infants and cognition in individuals with autism spectrum disorders and generating \$1.5M in funding for ILABS from organizations including the Simms Mann, Bezos Family, and Paros Foundations.}
      \item {I served in advisory roles with junior colleagues and students to provide technical expertise for planning data acquisition and analysis strategies for the timely delivery of data science products. In Joo, S. J., Tavabi, K., et al. (2021), I helped to develop strategies for data quality control, wrangling, and analysis resulting in a publication of findings on the neuronal correlates of reading proficiency in the \href{https://doi.org/10.1016/j.bandl.2020.104906}{journal Brain and Language}. In the publication by Clarke M. et al. (2020), I mentored a graduate student with Python script development and scientific writing of digital signal denoising or signal-to-noise enhancement methodology paper in the \href{https://doi.org/10.1016/j.jneumeth.2020.108700}{Journal of Neuroscience Methods}.}
      \item {\textbf{Technical Skills:} Experimentation, Digital Signal Processing (DSP), A/B testing, Analysis of Variance (ANOVA), general linear model, Case Control, Longitudinal Data, Exploratory Data Analysis (EDA), Principal Component Analysis (PCA), Dimensionality Reduction, Feature Engineering, Data visualization (R, Tidyverse), Data Mining, Extraction Transformation \& Loading (ETL), Data Modeling, Machine Learning, LOGIT, Random Forest, MATLAB, Python, Pandas SQL, Xarray, Scikit-learn, Pandas SQL, Linux, MacOS, Bash Scripting, Git.}
      \item {\textbf{Soft Skills:} Teamwork, Leadership, Time Management, Communication, Presentation skills, Grant Writing, Project Management, and Paradigm Development.}
    \end{cvitems}
  }

  %---------------------------------------------------------
  \cventry
  {Post-Doc Fellow} % Job title
  {The Children's Hospital of Philadelphia} % Organisation
  {Philadelphia, PA} % Location
  {Nov 2008 - Oct 2011} % Date(s) chktex 8
  {
    \begin{cvitems} % Description(s) of tasks/responsibilities
      \item {After one year of post-graduate training abroad, I joined The Department of Radiology Lurie Family Foundations MEG Imaging Center to design an experimental paradigm to examine language cognition in a National Institute for Health (R01-DC008871-02, Timothy Roberts Ph.D.) funded case-control study of 100+ school-aged participants with autism. The results from the paradigm were published in two NeuroReport journal articles (\href{https://journals.lww.com/00001756-201112070-00007}{i}, \href{https://journals.lww.com/00001756-201107130-00003}{ii}) and awarded a college loan repayment award (\$35k) by the NIH for outstanding translational research.}
      \item {The paradigm leveraged psycholinguistic parameters to localize neuronal correlates of semantic cognition in the brain. Due to its reliability, the exam was incorporated into the Radiology Department's standard battery of preoperative brain mapping exams for epilepsy treatment neurosurgery in pediatric patients.}
      \item {As a co-principal investigator on R01-DC008871-02, I implemented advanced non-parametric statistics to improve the reliability of hypothesis testing on large-array sensor data for autism case-control data resulting in a highly-cited publication in \href{https://doi.org/10.1016/j.biopsych.2011.01.015}{Biological Psychiatry}}. 
      \item {Planned and instructed accelerated applied research statistics seminar to facilitate data science research projects required for first-year medical school residents in the Radiology Department.}
      \item {\textbf{Technical Skills:} Case-Control, Mixed Linear Modeling, Analysis of Variance (ANOVA) MATLAB, Time-Frequency Analysis.}
      \item {\textbf{Soft Skills:} Course Planning, Presentation Skills, Scientific writing.}
    \end{cvitems}
  }

  %---------------------------------------------------------
  \cventry
  {Visiting Scientist} % Job title
  {MRC Cognition and Brain Sciences Unit} % Organisation
  {Cambridge, UK} % Location
  {Aug 2007 - Aug 2008} % Date(s) chktex 8
  {
    \begin{cvitems} % Description(s) of tasks/responsibilities
      \item {After completing my Doctoral degree in neuroscience, I joined the Cognition Brain Sciences Unit's newly established magnetoencephalography (MEG) lab to design an experimental paradigm to pilot data on speech perception and develop standard operating procedures for data acquisition.}
      \item {\textbf{Technical Skills:} MATLAB, Unix Cluster computing, Quality Control, Standard Operating Procedures.}
      \item {\textbf{Soft Skills:} Teamwork, Presentation Skills.}
    \end{cvitems}
  }

  %---------------------------------------------------------
\end{cventries}
