%-------------------------------------------------------------------------------
% CONFIGURATIONS
%-------------------------------------------------------------------------------
% A4 paper size by default, use 'letterpaper' for US letter
\documentclass[10.5pt, letterpaper]{russell}
% \usepackage[backend=biber,style=numeric,sorting=none]{biblatex}
% \addbibresource{} % Empty bib resource to prevent warning

% Configure page margins with geometry
\geometry{left=1.2cm, top=1.2cm, right=1.2cm, bottom=1.2cm, footskip=.5cm}

% Specify the location of the included fonts
\fontdir[fonts/]

% Color for highlights
% russell Colors: russell-emerald, russell-skyblue, russell-red, russell-pink, russell-orange
%                 russell-nephritis, russell-concrete, russell-darknight, russell-purple
\colorlet{russell}{russell-red}
% Uncomment if you would like to specify your own color
% \definecolor{russell}{HTML}{CA63A8}

% Colors for text
% Uncomment if you would like to specify your own color
% \definecolor{darktext}{HTML}{414141}
% \definecolor{text}{HTML}{333333}
% \definecolor{graytext}{HTML}{5D5D5D}
% \definecolor{lighttext}{HTML}{999999}

% Set false if you don't want to highlight section with russell color
\setbool{acvSectionColorHighlight}{true}

% If you would like to change the social information separator from a pipe (|) to something else
\renewcommand{\acvHeaderSocialSep}{\quad\infty\quad}


%-------------------------------------------------------------------------------
%	PERSONAL INFORMATION
%	Comment any of the lines below if they are not required
%-------------------------------------------------------------------------------
% Available options: circle|rectangle,edge/noedge,left/right
% \photo[rectangle,edge,right]{./examples/profile}
\name{Kambiz}{Tavabi Ph.D.}
\position{Principal Data Scientist{\enskip\cdotp\enskip}Neuroscience Expert}
\address{Seattle, WA.}

\mobile{206\cdotp719\cdotp3524}
\email{ktavabi@gmail.com}
% \dateofbirth{January 1st, 1970}
\homepage{ktavabi.github.io/}
% \github{github.com/ktavabi}
\linkedin{kambiz-tavabi-93255326b}
% \gitlab{gitlab-id}
%\stackoverflow{1634460}{kambiz}
% \twitter{@twit}
% \skype{skype-id}
% \reddit{reddit-id}
% \medium{madium-id}
% \kaggle{kambizio}
% \googlescholar{VO6NMiEAAAAJ\&hl}{Kambiz Tavabi}
%% \firstname and \lastname will be used
\googlescholar{VO6NMiEAAAAJ\&hl}{Kambiz Tavabi PhD}
% \extrainfo{extra information}

% \quote{``Simplicity is deceptively complicated."}

%-------------------------------------------------------------------------------
%	LETTER INFORMATION
%	All of the below lines must be filled out
%-------------------------------------------------------------------------------
% The company being applied to
\recipient
  {Meta Reality Labs - Wearables Voice AI Team}%
  {Redmond, WA}
% The date on the letter, default is the date of compilation
\letterdate{\today}
% The title of the letter
\lettertitle{AI Research Scientist, Language}
% How the letter is opened
\letteropening{\newline Dear Hiring Manager,}
% How the letter is closed
\letterclosing{Sincerely,}
% Any enclosures with the letter
%\letterenclosure[Attached]{Résumé}


%-------------------------------------------------------------------------------
\begin{document}

% Print the header with above personal information
% Give optional argument to change alignment(C: center, L: left, R: right)
\makecvheader[R]

% Print the footer with 3 arguments(<left>, <center>, <right>)
% Leave any of these blank if they are not needed
\makecvfooter
{\today}
{Kambiz Tavabi~~~·~~~Cover Letter}
{\thepage}

% Print the title with above letter information
\makelettertitle

%-------------------------------------------------------------------------------
%	LETTER CONTENT
%-------------------------------------------------------------------------------
\begin{cvletter}
  I am writing to express my strong interest in the AI Research Scientist, Language position with Meta's Wearables Voice AI team. The convergence of neuroscience and wearable AI represents one of the most exciting frontiers in technology today. As someone who has spent 15+ years decoding how the human brain processes speech and language, I am uniquely positioned to help Meta's team create revolutionary voice interfaces that truly understand and respond to natural human communication.

  \textbf{Bridging Human Speech Processing and AI Systems}

  My research journey began with a fundamental question: how does the human brain transform acoustic signals into meaningful language? This led me to develop sophisticated machine learning pipelines for analyzing speech perception, language comprehension, and neural responses to auditory stimuli. What I discovered through years of research at University of Washington and Children's Hospital of Philadelphia is directly applicable to the challenges facing wearable voice AI today.

  My doctoral work focused on phonological processing during speech perception, where I developed advanced digital signal processing algorithms to extract meaningful features from complex auditory neural signals. This experience translates directly to the core challenges in wearable voice AI: how to process multi-sensor audio input, extract relevant speech features, and optimize for natural voice interaction. My published research on spectral-temporal analysis of cortical oscillations during lexical processing provides insights into how humans naturally process speech in noisy, multi-modal environments—exactly the challenge faced by RayBan Smart Glasses and future wearable devices.

  \textbf{Scalable ML Systems and Research-to-Product Translation}

  At the Institute for Learning \& Brain Sciences, I built scalable Python-based ML pipelines that processed high-dimensional time-series data from multiple sensors simultaneously, developing novel approaches to feature extraction and pattern recognition that achieved real-time performance. This work included creating open-source tools (MNE-BIDS) now used internationally by researchers processing complex signal data—demonstrating my ability to develop production-ready systems that scale across teams and applications.

  My current role as Principal Data Scientist has strengthened my ability to translate cutting-edge research into practical applications. I developed and deployed a production ML system using TF-IDF vectorization and advanced classification algorithms that processes narrative text data for 27,000+ individuals annually. This experience taught me how to bridge the gap between research innovation and scalable, reliable systems—a critical skill for advancing device-driven AI assistants from research to product.

  \textbf{Technical Excellence in Modern AI/ML}

  My technical foundation strongly aligns with Meta's requirements:

  \begin{itemize}
    \item \textbf{Speech \& Language Processing}: Digital signal processing, audio signal analysis, speech recognition, natural language processing, and multi-sensor data fusion
    \item \textbf{Machine Learning}: Neural networks, deep learning frameworks (PyTorch, TensorFlow), statistical learning, and large-scale model training
    \item \textbf{Programming \& Development}: Python (Jupyter, Pandas, NumPy, SciPy, Scikit-learn), R, MATLAB, SQL, and Git/GitHub
    \item \textbf{Research Translation}: Experimental design, statistical inference, and proven ability to move from research concepts to production systems
  \end{itemize}

  While my foundational expertise lies in traditional ML and statistical approaches, I have been actively expanding my knowledge in large language models through recent coursework in Bayesian statistics and advanced Python applications. My strong mathematical foundation, combined with extensive experience in neural network architectures and signal processing, positions me to quickly contribute to state-of-the-art LLM development.

  \textbf{Vision for Wearables Voice AI}

  What excites me most about this opportunity is the chance to apply my deep understanding of human speech and language processing to create AI systems that interact as naturally as humans do. My research on early language development and speech discrimination has revealed key insights about how humans process speech in challenging acoustic environments, handle multiple simultaneous inputs, and extract meaning from context—all critical capabilities for next-generation wearable AI.

  I see tremendous potential in combining my neuroscience background with Meta's AI capabilities to create voice interfaces that don't just recognize speech, but truly understand human communication patterns. My experience with multi-sensor data fusion, real-time signal processing, and understanding the neural basis of language comprehension could help develop wearable AI that adapts to individual users' communication styles, processes speech in noisy real-world environments, and provides contextually appropriate responses.

  \textbf{Collaborative Research Excellence}

  I am particularly drawn to the collaborative nature of this role, working closely with Meta GenAI and FAIR teams. My track record of successful interdisciplinary collaboration—from mentoring graduate students to managing cross-functional research teams and \$2M+ in research grants—demonstrates my ability to contribute effectively to Meta's innovative research environment while helping translate breakthrough discoveries into transformative products.

  The opportunity to shape the future of human-AI interaction through wearable devices represents the perfect convergence of my research expertise and passion for practical innovation. I would welcome the opportunity to discuss how my unique background in speech and language processing can contribute to Meta's vision for the next evolution in social technology.

  Thank you for considering my application. I look forward to the possibility of contributing to Meta's groundbreaking work in wearables voice AI.

\end{cvletter}

%-------------------------------------------------------------------------------
% Print the signature and enclosures with above letter information
\makeletterclosing

\end{document}
