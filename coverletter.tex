%-------------------------------------------------------------------------------
% CONFIGURATIONS
%-------------------------------------------------------------------------------
% A4 paper size by default, use 'letterpaper' for US letter
\documentclass[10.5pt, letterpaper]{russell}
% \usepackage[backend=biber,style=numeric,sorting=none]{biblatex}
% \addbibresource{} % Empty bib resource to prevent warning

% Configure page margins with geometry
\geometry{left=1.2cm, top=1.2cm, right=1.2cm, bottom=1.2cm, footskip=.5cm}

% Specify the location of the included fonts
\fontdir[fonts/]

% Color for highlights
% russell Colors: russell-emerald, russell-skyblue, russell-red, russell-pink, russell-orange
%                 russell-nephritis, russell-concrete, russell-darknight, russell-purple
\colorlet{russell}{russell-red}
% Uncomment if you would like to specify your own color
% \definecolor{russell}{HTML}{CA63A8}

% Colors for text
% Uncomment if you would like to specify your own color
% \definecolor{darktext}{HTML}{414141}
% \definecolor{text}{HTML}{333333}
% \definecolor{graytext}{HTML}{5D5D5D}
% \definecolor{lighttext}{HTML}{999999}

% Set false if you don't want to highlight section with russell color
\setbool{acvSectionColorHighlight}{true}

% If you would like to change the social information separator from a pipe (|) to something else
\renewcommand{\acvHeaderSocialSep}{\quad\infty\quad}


%-------------------------------------------------------------------------------
%	PERSONAL INFORMATION
%	Comment any of the lines below if they are not required
%-------------------------------------------------------------------------------
% Available options: circle|rectangle,edge/noedge,left/right
% \photo[rectangle,edge,right]{./examples/profile}
\name{Kambiz}{Tavabi Ph.D.}
\position{Principal Data Scientist{\enskip\cdotp\enskip}Neuroscience Expert}
\address{Seattle, WA.}

\mobile{206\cdotp719\cdotp3524}
\email{ktavabi@gmail.com}
% \dateofbirth{January 1st, 1970}
\homepage{ktavabi.github.io/}
% \github{github.com/ktavabi}
\linkedin{kambiz-tavabi-93255326b}
% \gitlab{gitlab-id}
%\stackoverflow{1634460}{kambiz}
% \twitter{@twit}
% \skype{skype-id}
% \reddit{reddit-id}
% \medium{madium-id}
% \kaggle{kambizio}
% \googlescholar{VO6NMiEAAAAJ\&hl}{Kambiz Tavabi}
%% \firstname and \lastname will be used
\googlescholar{VO6NMiEAAAAJ\&hl}{Kambiz Tavabi}
% \extrainfo{extra information}

% \quote{``Simplicity is deceptively complicated."}

%-------------------------------------------------------------------------------
%	LETTER INFORMATION
%	All of the below lines must be filled out
%-------------------------------------------------------------------------------
% The company being applied to
\recipient
  {Fred Hutchinson Cancer Center}%
  {1100 Fairview Ave N\\Seattle, WA 98109}
% The date on the letter, default is the date of compilation
\letterdate{\today}
% The title of the letter
\lettertitle{Data Manager - Translational Research Program in Colorectal Cancer Disparities}
% How the letter is opened
\letteropening{\newline Dear Hiring Manager,}
% How the letter is closed
\letterclosing{Sincerely,}
% Any enclosures with the letter
%\letterenclosure[Attached]{Résumé}


%-------------------------------------------------------------------------------
\begin{document}

% Print the header with above personal information
% Give optional argument to change alignment(C: center, L: left, R: right)
\makecvheader[R]

% Print the footer with 3 arguments(<left>, <center>, <right>)
% Leave any of these blank if they are not needed
\makecvfooter
{\today}
{Kambiz Tavabi~~~·~~~Cover Letter}
{\thepage}

% Print the title with above letter information
\makelettertitle

%-------------------------------------------------------------------------------
%	LETTER CONTENT
%-------------------------------------------------------------------------------
\begin{cvletter}
  I am writing to express my strong interest in the Data Manager position (Job ID 29498) supporting the Translational Research Program in Colorectal Cancer Disparities (TRPCD) at Fred Hutchinson Cancer Center. With over 15 years of experience managing complex research datasets and coordinating multi-institutional collaborations, I am excited about the opportunity to contribute to Fred Hutch's mission of advancing cancer research and treatment.

  My background aligns perfectly with the technical and organizational requirements of this role. As Principal Data Scientist at the Washington State Department of Corrections, I currently manage longitudinal datasets with 27,000+ individual cases annually, implementing standardized analytical frameworks and quality control measures to ensure data integrity. I oversee development of scalable data science infrastructure using Python, R, and SQL databases, creating automated workflows for reproducible research that directly support executive decision-making.

  During my 12-year tenure as Research Science Engineer at the University of Washington's Institute for Learning \& Brain Sciences, I managed complex multimodal datasets from 500+ participants across multiple research projects, collaborating extensively with biostatisticians, clinicians, and laboratory personnel. I collaborated with the development of MNE-BIDS, an internationally-adopted Python package for organizing and analyzing biomedical data, which demonstrates my expertise in creating standardized workflows for data sharing and collaboration—a skill directly relevant to the TRPCD's multi-institutional projects.

  My experience in clinical data coordination is particularly relevant to this position. At The Children's Hospital of Philadelphia, I managed large case-control longitudinal datasets for pediatric research cohorts, developing protocols for data collection, annotation, and integration. I have extensive experience implementing quality control protocols, data validation procedures, and maintaining comprehensive data dictionaries and codebooks for reproducible research.

  Key qualifications that make me an excellent fit for this role include:

  \textbf{Data Management \& Analysis:} Proficiency in Python and R programming, with extensive experience processing large-scale datasets and implementing automated data cleaning and analysis pipelines. I have managed complex longitudinal datasets and developed quality control measures to ensure data integrity.

  \textbf{Multi-institutional Collaboration:} Proven track record coordinating across multiple institutions, cores, and external collaborators. I have managed research grants totaling \$2M+ and facilitated data sharing among diverse research teams.

  \textbf{Documentation \& Compliance:} Experience maintaining comprehensive documentation using GitHub version control and developing standard operating procedures (SOPs) for data management. I have ensured compliance with institutional standards and implemented quality control measures.

  \textbf{Technical Operations:} Expertise in database systems, cloud computing, and scientific computing software including Python, SQL, MATLAB, and statistical software (R, SPSS, jamovi). I have experience with spreadsheets and maintaining data backup and security protocols.

  \textbf{Communication \& Training:} Strong ability to translate complex data-related issues for non-technical stakeholders and train team members on data management procedures. I have presented data updates to investigators and prepared progress reports for funding agencies.

  I am particularly drawn to Fred Hutch's commitment to collaboration, innovation, and excellence in biomedical research. The opportunity to support the TRPCD's work on colorectal cancer disparities aligns with my passion for using data science to address critical health challenges. My experience managing large-scale, multi-institutional datasets and coordinating with diverse stakeholders would enable me to contribute immediately to the program's research objectives.

  I would welcome the opportunity to discuss how my background in data management, multi-institutional collaboration, and research coordination can support the TRPCD's mission. Thank you for considering my application.

\end{cvletter}

%-------------------------------------------------------------------------------
% Print the signature and enclosures with above letter information
\makeletterclosing

\end{document}
