%-------------------------------------------------------------------------------
% CONFIGURATIONS
%-------------------------------------------------------------------------------
% A4 paper size by default, use 'letterpaper' for US letter
\documentclass[13pt, letterpaper]{russell}

% Configure page margins with geometry
\geometry{left=1.4cm, top=.8cm, right=1.4cm, bottom=1.8cm, footskip=.5cm}

% Specify the location of the included fonts
\fontdir[fonts/]

% Color for highlights
% russell Colors: russell-emerald, russell-skyblue, russell-red, russell-pink, russell-orange
%                 russell-nephritis, russell-concrete, russell-darknight, russell-purple
\colorlet{russell}{russell-black}
% Uncomment if you would like to specify your own color
% \definecolor{russell}{HTML}{CA63A8}

% Colors for text
% Uncomment if you would like to specify your own color
% \definecolor{darktext}{HTML}{414141}
% \definecolor{text}{HTML}{333333}
% \definecolor{graytext}{HTML}{5D5D5D}
% \definecolor{lighttext}{HTML}{999999}

% Set false if you don't want to highlight section with russell color
\setbool{acvSectionColorHighlight}{true}

% If you would like to change the social information separator from a pipe (|) to something else
\renewcommand{\acvHeaderSocialSep}{\quad\infty\quad}


%-------------------------------------------------------------------------------
%	PERSONAL INFORMATION
%	Comment any of the lines below if they are not required
%-------------------------------------------------------------------------------
% Available options: circle|rectangle,edge/noedge,left/right
% \photo[rectangle,edge,right]{./examples/profile}
\name{Kambiz}{Tavabi, PhD}
% \position{Software Architect{\enskip\cdotp\enskip}Security Expert}
\address{Seattle, WA.}

\mobile{206\cdotp719\cdotp3524}
\email{ktavabi@gmail.com}
%\dateofbirth{January 1st, 1970}
%\homepage{www.posquit0.com}
\github{github.com/ktavabi}
% \linkedin{linkedin.com/in/themagicalmammal}
% \gitlab{gitlab-id}
\stackoverflow{1634460}{kambiz}
% \twitter{@twit}
% \skype{skype-id}
% \reddit{reddit-id}
% \medium{madium-id}
% \kaggle{kaggle-id}
% \googlescholar{googlescholar-id}{name-to-display}
%% \firstname and \lastname will be used
\googlescholar{VO6NMiEAAAAJ\&hl}{}
% \extrainfo{extra information}

% \quote{``Simplicity is deceptively complicated."}

%-------------------------------------------------------------------------------
%	LETTER INFORMATION
%	All of the below lines must be filled out
%-------------------------------------------------------------------------------
% The company being applied to
\recipient
  {} % FIXME
  {Dept. of Social and Health Services} % FIXME
% The date on the letter, default is the date of compilation
\letterdate{\today}
% The title of the letter
\lettertitle{IT Statistical Programmer Job No. 12261} % FIXME
% How the letter is opened
\letteropening{\newline Greetings,}
% How the letter is closed
\letterclosing{Sincerely,}
% Any enclosures with the letter
\letterenclosure[Attached]{Résumé}


%-------------------------------------------------------------------------------
\begin{document}

% Print the header with above personal information
% Give optional argument to change alignment(C: center, L: left, R: right)
\makecvheader[R]

% Print the footer with 3 arguments(<left>, <center>, <right>)
% Leave any of these blank if they are not needed
\makecvfooter
{\today}
{Kambiz Tavabi~~~·~~~Cover Letter}
{\thepage}

% Print the title with above letter information
\makelettertitle

%-------------------------------------------------------------------------------
%	LETTER CONTENT
%-------------------------------------------------------------------------------
\begin{cvletter}
  I am writing to express my interest in the IT Statistical Programmer Job No. 12261 or machine learning \& ethical AI opportunities in the Facilities, Finance, and Analytics department. I have over thirteen years of experience managing both Federally \& privately funded biomedical data science projects in academia. I received my Ph.D. in neuroscience from Münster, Germany (2007), with a thesis on speech perception using neuropsychology experimentation \& biomedical imaging technologies to publish two first-authored articles for my dissertation. Before that, as a volunteer student, I began my journey with data using a no. 2 pencil and a microscope in a neuroanatomy lab at UCLA (1999). So I've been around data for over half my life! I am a thoughtful and tactically nimble worker, capable of adapting to work with different team sizes or independently. I am interested in the IT Statistical Programmer because social health data is in my expertise domain. It also offers an opportunity for me to transfer my data science skills into a space with a more immediate impact on social health policy in the WA state government.

 I have overseen the strategic planning of budgets, federal and private grants, paradigms, and timely communication of deliverables to stakeholders. My academic research is at the intersection of neuroscience and data science. And if I were to choose one lesson about data from my experience, it is that the road from big data to discovery is a two-way street. One is constantly learning the evolving tools used to work with data; the other is the drive to curate knowledge from information or signal extracted from noise. As a UW staff scientist, I managed several end-to-end data science projects. I created data acquisition, ETL, \& wrangling pipelines for imaging and dense-array health data. My tool of choice is Python because of my work culture and because I find it more intuitive than R or MATLAB. Despite the prevailing culture, I have learned to use R because it offers a deeper insight into the calculus of statistical modeling.

 As a data scientist, I have used Python to develop standardized workflows for faster and more reliably reproducible data processing pipelines and used machine learning with Scikit-Learn to implement classification (LOGIT, Regression) and feature selection (random forest) for case-control and A/B datasets. However, I am keen and more than willing to supplant my python skills with SAS or R programming to develop the knowledge curation pipelines required within your organization. The fact is that being around data for as long as I have makes it easy to be data agnostic. I believe that the IT Statistical Programmer or similar data science roles in DSHS is an excellent opportunity to become a more solid applied data science practitioner. 
 
 I look forward to hearing from you to discuss how I can best meet your data needs. Thank you for your time and consideration.

\end{cvletter}

%-------------------------------------------------------------------------------
% Print the signature and enclosures with above letter information
\makeletterclosing

\end{document}
